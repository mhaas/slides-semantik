\documentclass[12pt,a4paper]{beamer}
\usepackage[utf8x]{inputenc}
\usepackage{ucs}
\usepackage[english]{babel}
%\usepackage[german]{babel}
\usepackage{amsmath}
\usepackage{amsfonts}
\usepackage{amssymb}
\author{Michael Haas, haas@cl.uni-heidelberg.de}
\title{Rudolph \& Giesbrecht: Compositional Matrix-Space Models of Language}
\subtitle{Seminar: Distributionelle Semantik jenseits der Wortbedeutung (Matthias Hartung)}
\date{24-06-2013}
\begin{document}


\begin{frame}
\maketitle
\end{frame}

\begin{frame}{Overview}
\begin{itemize}
\item Idea \& Approach %- Dist. Semantics, PSS, Perceptual Simulation
\item Background PSS
\item Data
\item Evaluation
    \begin{itemize}
    \item Model Foundations
    \item Behavioral Simulations
    \end{itemize}
\item Recap \& Discussion
\item \textbf{Questions? Too fast? Ask!}
\end{itemize}
\end{frame}


%\begin{frame}{Perceptual Systems Theory: Amodal theories}
%\begin{figure}
%\includegraphics[scale=0.8]{barsalou_figure_2_amodal_symbol_systems.png}
%\caption{Basic idea of Amodal theories, \cite{barsalou}}
%\end{figure}
%\end{frame}

\begin{frame}[allowframebreaks]{References}
\begin{thebibliography}{-}
% APA
\bibitem{cmsm} Rudolph, S., \& Giesbrecht, E. (2010, July). Compositional matrix-space models of language. In Proceedings of the 48th Annual Meeting of the Association for Computational Linguistics (pp. 907-916). Association for Computational Linguistics.
\bibitem{cmsmse} Yessenalina, A., \& Cardie, C. (2011, July). Compositional matrix-space models for sentiment analysis. In Proceedings of the Conference on Empirical Methods in Natural Language Processing (pp. 172-182). Association for Computational Linguistics.
\end{thebibliography}
\end{frame}
\end{document}
