\documentclass[12pt,a4paper]{beamer}
\usepackage[utf8x]{inputenc}
\usepackage{ucs}
\usepackage[english]{babel}
%\usepackage[german]{babel}
\usepackage{amsmath}
\usepackage{amsfonts}
\usepackage{amssymb}
\author{Michael Haas, haas@cl.uni-heidelberg.de}
\title{Rudolph \& Giesbrecht: Compositional Matrix-Space Models of Language}
\subtitle{Seminar: Distributionelle Semantik jenseits der Wortbedeutung (Matthias Hartung)}
\date{24-06-2013}
\begin{document}


\begin{frame}
\maketitle
\end{frame}

\begin{frame}{Overview}
\begin{itemize}
\item Previous approaches % incl problems
\item Idea \& Approach % incl motivation, or see prev?
\item Foundations
\begin{itemize}
    \item Plausibility % grounding in other theories
    \item Encoding VSM % + examples
    \item Encoding symbolic approaches
\end{itemize}
\item A first implementation: Sentiment Analysis
\begin{itemize}
    \item Algorithm % cursory
    \item Results % cursory - was baseline sensible? what is state of the art?
\end{itemize}
\item Recap \& Discussion
\item \textbf{Questions? Too fast? Ask!}
\end{itemize}
\end{frame}


\begin{frame}{Foundations} % + justification?
\begin{itemize}
\item Plausibility in various systems
\item CMSMs can represent various VSM
\item CMSMs can represent formal languages
\end{itemize}
\end{frame}

\begin{frame}{Foundations: Plausibility: Algebraic Properties}
\begin{itemize}
\item Traditional VSM (e.g. Vector Mixture (Lapata, Mitchell, 2010??) 
 use associate and commutative operators
\item $\to$ Commutativity ignores word order
\item Matrix multiplication is \textbf{non-commutative}
\end{itemize}
\end{frame}

\begin{frame}{Foundations: Plausibility: Neurological}
\begin{itemize}
\item Mental state: vector of neuron excitation values
\item Word as stimulus transforms mental state
\item $\to$ word is a function mapping mental states to mental states
\item Arbitrary length transformations realised through matrix multiplication
\end{itemize}
\end{frame}



\begin{frame}{Foundations}
\begin{itemize}
\item
\end{itemize}
\end{frame}


%\begin{frame}{Perceptual Systems Theory: Amodal theories}
%\begin{figure}
%\includegraphics[scale=0.8]{barsalou_figure_2_amodal_symbol_systems.png}
%\caption{Basic idea of Amodal theories, \cite{barsalou}}
%\end{figure}
%\end{frame}

\begin{frame}[allowframebreaks]{References}
\begin{thebibliography}{-}
% APA
\bibitem{cmsm} Rudolph, S., \& Giesbrecht, E. (2010, July). Compositional matrix-space models of language. In Proceedings of the 48th Annual Meeting of the Association for Computational Linguistics (pp. 907-916). Association for Computational Linguistics.
\bibitem{cmsmse} Yessenalina, A., \& Cardie, C. (2011, July). Compositional matrix-space models for sentiment analysis. In Proceedings of the Conference on Empirical Methods in Natural Language Processing (pp. 172-182). Association for Computational Linguistics.
\end{thebibliography}
\end{frame}
\end{document}
